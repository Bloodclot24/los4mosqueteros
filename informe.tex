\documentclass[a4paper,10pt,titlepage]{article}
\usepackage[left=2cm,top=3cm,right=2cm,bottom=1cm,head=2.0cm,includefoot]{geometry}
\usepackage[utf8]{inputenc}
\usepackage[spanish]{babel}
\usepackage{comment}
\usepackage{fancyhdr}
\usepackage[T1]{fontenc}
\usepackage{graphicx}
\usepackage{bookman}
\usepackage{amsmath}
\usepackage{color}
\usepackage{listings}
\usepackage{longtable}
\usepackage{moreverb}
\usepackage{booktabs}
\usepackage{multirow}
\usepackage{ulem}
\usepackage[pdfborder={0 0 0 0}]{hyperref}
\usepackage{fixltx2e}
\usepackage{float}
\usepackage{wrapfig}
\usepackage{soul}
\usepackage{t1enc}
\usepackage{textcomp}
\usepackage{marvosym}
\usepackage{wasysym}
\usepackage{latexsym}
\usepackage{amssymb}
\usepackage{microtype}
\usepackage{hyperref}
\usepackage{pdfpages} % to import PDF pages
\tolerance=1000


\title{75.15 - Base de Datos}
\author{\textbf{Grupo 2}}

\begin{document}

\pagestyle{fancy}
\chead{75.15 - Base de Datos}
\lhead{\includegraphics[width=1.5cm]{./logo1.png}}
\lfoot{Grupo 2}
\rfoot{$2^{do}$ Cuatrimestre 2010}
\maketitle

\tableofcontents
\newpage

%documento
\section{Enunciado}
El canal de televisi\'on por cable ALAMBRIN, mantiene informaci\'on de la programaci\'on, publicidad y producci\'on
de sus programas. Se cuenta con la siguiente informaci\'on:\\
\\ \textbf{Programaci\'on} \\
La programaci\'on del canal se emite diariamente de 7 a 24 hs. En ese horario se emiten programas en vivo o
programas grabados. Cada programa tiene un nombre que lo identifica y una duraci\'on determinada. El canal
cuenta con distintas clases de programas: Dibujos animados, Juegos y Entretenimientos, Humorísticos,
Noticieros infantiles y Películas infantiles.
El horario de transmisi\'on del canal se divide en espacios que tienen una duraci\'on determinada, identificados por
el día y la hora en que se emite, y caracterizado por un precio ( por segundo en el aire ).
El departamento de Programaci\'on del canal asigna a cada espacio un programa, y determina adem\'as si el
espacio contiene tandas publicitarias, identificadas por el día, la hora y el n\'umero de tanda. Dichas tandas se
corresponden a un espacio específico.\\
\\ \textbf{Publicidad}\\
El canal cuenta con una serie de anunciantes. A cada uno se le asigna un n\'umero de cuenta, y para cada uno se
cuenta con su raz\'on social, domicilio y tel\'efono.
En cada tanda se publicitan varios anuncios, identificados por un título. Cada anuncio se caracteriza por el
producto ofrecido, la duraci\'on del mismo, y el anunciante que lo auspicia.
Es importante para los anunciantes el orden en que saldr\'a al aire el anuncio dentro de la tanda, ya que de esa
forma obtendr\'an mayor atenci\'on de los televidentes. Como ALAMBRIN es el canal líder para los niños, muchos
anuncios salen al aire en diferentes tandas publicitarias.\\
\\ \textbf{Producci\'on}\\
Existen productores independientes que pueden proveer programas grabados al canal. Para aquellos
productores ligados al canal se conoce su n\'umero de CUIT y raz\'on social.
Debido al nuevo cambio de imagen propuesto por el departamento de Marketing, ALAMBRIN ha invertido en
m\'oviles en exteriores de diferentes características t\'ecnicas. Dichos m\'oviles se utilizan en diferentes programas
en vivo. Pero no todos los programas en vivo utilizan m\'oviles en exteriores. Algunos programas en vivo suelen
tener asignados m\'as de un m\'ovil, sobre todo en \'epoca de vacaciones.
Otro de los cambios de imagen impulsados por el canal consta en hacer participar al p\'ublico infantil de los
programas en vivo. La cantidad de p\'ublico que ingresa al canal depende de la capacidad de los estudios
ocupados para producir el programa. Cada estudio se identifica por un n\'umero \'unico.\\
\\
 \textbf{Hip\'otesis}
\begin{itemize}
\item En una misma tanda no puede repetirse un anuncio.
\item La suma de las duraciones de los anuncios de una tanda, determina la duraci\'on de esa tanda.
\item La suma de las duraciones de las tandas que componen un espacio, junto con la duraci\'on del programa
emitido en dicho espacio, coincide exactamente con la duraci\'on del espacio
\end{itemize}

\subsection{Objetivo del Trabajo Pr\'actico}

\begin{enumerate}
\item Realizar un modelo E-R en base a la descripci\'on del enunciado y a los efectos de satisfacer los
requerimientos de informaci\'on solicitados.
\item Transformar el modelo E-R en un modelo relacional (modelo de tablas) utilizando los conocimientos de
transformaci\'on de entidades a tablas.
\end{enumerate}

\textbf{Forma De Presentaci\'on Del Trabajo Practico:}
\begin{enumerate}
\item Presentar el diagrama de entidad - interrelaci\'on con indicaciones de restricciones de cardinalidad.
\item Indicar dependencias de identidad y de existencia en el modelo.
\item Especificar supuestos que justifiquen el modelo (Hip\'otesis).
\item Presentar el diccionario de datos del diagrama con la siguiente informaci\'on:\\
Para cada tipo de entidad se debe especificar:\\
a) Definici\'on.\\
b) Especificaci\'on de atributos.\\
c) Especificaci\'on de identificador \'unico.\\
Para cada tipo de interrelaci\'on se debe especificar:\\
a) Definici\'on.\\
b) Especificaci\'on de atributos.\\
c) Especificaci\'on de identificador \'unico.\\

\item Presentar el modelo Relacional ( "de tablas" ) indicando para cada esquema de relaci\'on:
\begin{itemize}
\item Atributos
\item Claves candidatas
\item Clave primaria
\item Claves for\'aneas
\item Atributos que pueden tomar valores nulos
\item Realice el diagrama del Modelo de Tablas
\end{itemize}

\textbf{Nota}: en los casos en que existan diferentes alternativas para efectuar la transformaci\'on de MER al modelo
de tablas, elegir una \'unica alternativa y enumerar las ventajas y desventajas de la alternativa elegida.

\end{enumerate}

\pagebreak
\section{Resoluci\'on}
\subsection{Diagrama Entidad - Interrelaci\'on}


\pagebreak

\subsection{Hip\'otesis}
\begin{itemize}
\item Un programa tiene asociado a lo sumo un estudio.
\item Cada m\'ovil de exterior se identifica un\'ivocamente por un n\'umero de m\'ovil.
\item Cada programa tiene asociado un espacio (no existen programas sin asociar).
\item Un programa tiene s\'olo un productor; un productor puede producir varios programas.
\end{itemize}

\subsection{Dependencias}

Se identific\'o la entidad d\'ebil Tanda, la cual depende de la entidad Espacio.\\
Para identificar a la entidad Tanda, utilizamos, el dia y la hora que son clave de la entidad Espacio y como atributo discriminante de la entidad d\'ebil usamos NroTanda.

\pagebreak

\subsection{Diccionario}
\subsubsection{Entidades}
\begin{enumerate}

\item {\bf Espacio}

\begin{itemize}
 
\item \underline{Definici\'on}

El tiempo de transmici\'on se divide en espacios. Esta entidad, representa cada uno de dichos espacios.
Dentro de estos puede haber tandas y un programa. Los espacios cuentan con una duraci\'on
(la misma se obtiene como: la suma de las duraciones de las tandas que contiene y la duraci\'on del programa), una fecha, una hora y un precio determinados.

\item \underline{Especificaci\'on de atributos}

-Dia: Especifica el d\'ia en que se ubica el espacio dentro de la programaci\'on.\\
-Fecha: Especifica la fecha en la cu\'al va a emitirse el espacio en cuestion.\\
-Hora: Especifica la hora de inicio del espacio en cuestion.\\
-Precio: Especifica el valor comercial del espacio.\\

\item \underline{Especificaci\'on de identificador \'unico}\\
-Dia: \\
-Hora\\
\end{itemize}

\item {\bf Programa}

\begin{itemize}
 
\item \underline{Definici\'on}

Un programa est\'a contenido en un espacio. 

\item \underline{Especificaci\'on de atributos}

-Tipo: Este atributo especifica el tipo del programa en cuestion, el mismo puede ser: dibujos animados,juegos y entretenimientos,humor\'isticos,noticieros infantiles y pel\'iculas infantiles.\\
-Duracion: Representa la duraci\'on total del programa.

\item \underline{Especificaci\'on de indentificador \'unico}

-Nombre: Representa el nombre del programa en cuestion.

\end{itemize}

\item {\bf Programa en vivo}
\begin{itemize}
 
\item \underline{Definici\'on}

Un programa en vivo est\'a contenido en un espacio y tiene moviles asociados. Es una especificaci\'on de un Programa. 

\item \underline{Especificaci\'on de atributos}

No tiene atributos propios.

\item \underline{Especificaci\'on de indentificador \'unico}

-Nombre: Representa el nombre del programa en cuestion.

\end{itemize}

\item {\bf Programa grabado}
\begin{itemize}
 
\item \underline{Definici\'on}

Un programa grabado est\'a contenido en un espacio y tiene productores independientes asociados. Es una especificaci\'on de Programa. 

\item \underline{Especificaci\'on de atributos}\\
No tiene atributos propios

\item \underline{Especificaci\'on de indentificador \'unico}

-Nombre: Representa el nombre del programa en cuestion.

\end{itemize}

\item {\bf M\'ovil Exterior}

\begin{itemize}
 
\item \underline{Definici\'on}

Esta entidad representa el m\'ovil de exterior que puede estar asociado a un programa en vivo.

\item \underline{Especificaci\'on de identificador \'unico}

-NroMovil: Representa el n\'umero que identifica un\'ivocamente a cada uno de los moviles de exterior.

\end{itemize}

\item {\bf Estudio}

\begin{itemize}

\item \underline{Definici\'on}

Representa el lugar f\'isico en el cual se desarrolla un programa en vivo.

\item \underline{Especificaci\'on de atributos}

-CantidadPersonas: Representa la cantidad m\'axima de personas que puede alojar un estudio.

\item \underline{Especificaci\'on de identificador \'unico}

-NroEstudio: Representa el n\'umero que identifica un\'ivocamente cada estudio.

\end{itemize}


\item {\bf Productor Independiente}

\begin{itemize}

\item \underline{Definici\'on}

Representa al productor que le provee al canal sus programas grabados.

\item \underline{Especificaci\'on de atributos}

-Razon social: Representa la raz\'on social del productor en cuestion.

\item \underline{Especificaci\'on de identificador \'unico}

-CUIT: Representa el n\'umero de CUIT del productor en cuestion.

\end{itemize}


\item {\bf Anuncio}

\begin{itemize}

\item \underline{Definici\'on}

Es la publicidad que representa a un producto en particular.

\item \underline{Especificaci\'on de atributos}

-Duracion: Representa la duraci\'on del anuncio a emitir.\\
-Producto ofrecido: Representa el producto que se ofrece en el anuncio.

\item \underline{Especificaci\'on de identificador unico}

-Titulo: Representa el t\'itulo del anuncio en cuestion.

\end{itemize}


\item {\bf Anunciante}

\begin{itemize}

\item \underline{Definici\'on}

Es la entidad propietaria de un anuncio en particular. 

\item \underline{Especificaci\'on de atributos}

-Razon social: Representa la raz\'on social del anunciante.\\
-Telefono: Representa el tel\'efono del anunciante.\\
-Domicilio: Representa la direcci\'on f\'isica o legal del anunciante.\\

\item \underline{Especificaci\'on de identificador \'unico}

-NroCuenta: Representa el n\'umero de la cuenta que el anunciante tiene con el canal.

\end{itemize}
\end{enumerate}

\subsubsection{Interrelaciones}

\textbf{1- Se publicita en:}
\begin{itemize}
\item \underline{Definici\'on}
Relaciona un Anuncio con una Tanda, indicando el orden en que ese Anuncio es emitido dentro de una Tanda en particular.
\item \underline{Especificaci\'on de atributos}
-Orden: Establece el numero de orden en que un Anuncio es emitido en una Tanda. 
\item \underline{Especificaci\'on de identificador \'unico}
- Orden, tituloAnuncio, dia, hora, nroTanda: La publicidad se realiza en un dia y hora pertenecientes a la Tanda, en orden espec\'ifico para cada Anuncio.
\end{itemize}

-\textbf{Nota}: El resto de las interrelaciones se considera que no necesitan especificaci\'on ya que simplemente relacionan entidades, sin ninguna particularidad. 

\pagebreak
\subsection{Modelo Relacional}

\subsubsection{Dise\~no del Modelo}

PROGRAMA (nombrePrograma, duracion, tipo)\\
PK: (nombrePrograma)\\
FK: --\\
CK: (nombrePrograma)\\
NV: --\\
\\
PROGRAMA\_VIVO (nombrePrograma, nroEstudio)\\
PK: (nombrePrograma)\\
FK: (nroEstudio)\\
CK: (nombrePrograma)\\
NV: (nroEstudio)\\
\\
PROGRAMA\_GRAB (nombrePrograma, cuit\_pi)\\
PK: (nombrePrograma)\\
FK: (cuit\_pi)\\
CK: (nombrePrograma)\\
NV: --\\
\\
ESPACIO (dia, hora, duracion, precio, nombrePrograma)\\
PK: (dia, hora)\\
FK: (nombrePrograma)\\
CK: (dia, hora)\\
NV: (precio)\\
\\
TANDA (nroTanda, dia, hora)\\
PK: (nroTanda)\\
FK: (dia, hora)\\
CK: (nroTanda)\\
NV: --\\
\\
SE\_PUBLICITA\_EN (nroTanda, dia, hora, tituloAnuncio, orden)\\
PK: (orden, dia, hora, tituloAnuncio, nroTanda)\\
FK: (nroTanda, dia, hora); (tituloAnuncio)\\
CK: (orden, dia, hora, tituloAnuncio, nroTanda)\\
NV: --\\
\\
ANUNCIO (tituloAnuncio, productoOfrecido, duracion, nroCuenta)\\
PK: (tituloAnuncio)\\
FK: (nroCuenta)\\
CK: (tituloAnuncio)\\
NV: --\\
\\
ANUNCIANTE (nroCuenta, razonSocial, domicilio, telefono)\\
PK: (nroCuenta)\\
FK: --\\
CK: (nroCuenta); (razonSocial)\\
NV: (domicilio, telefono)\\
\\
UTILIZA\_UN (nombrePrograma, nroMovil)\\
PK: (nombrePrograma, nroMovil)\\
FK: (nombrePrograma); (nroMovil)\\
CK: (nombrePrograma, nroMovil)\\
NV: --\\
\\
MOVIL\_EXTERIOR (nroMovil)\\
PK: (nroMovil)\\
FK: --\\
CK: (nroMovil)\\
NV: --\\
\\
ESTUDIO (nroEstudio, cantidadPersonas)\\
PK: (nroEstudio)\\
FK: --\\
CK: (nroEstudio)\\
NV: (cantidadPersonas)\\
\\
PRODUCTOR\_INDEPENDIENTE (cuit\_pi, razonSocial)\\
PK: (cuit\_pi)\\
FK: --\\
CK: (cuit\_pi); (razonSocial)\\
NV: --\\
\\
\underline{NOTACION:}\\
PK: Clave Primaria o Primary Key\\
FK: Clave Foranea o Foreign Key\\
CK: Clave Candidata o Candidate Key\\
NV: Atributos que pueden tomar valores nulos o Null Values\\
\\

\subsubsection{Diagrama de Tablas}
\begin{center}
\includegraphics[width=18cm]{./tablas.png}    
\end{center}

\end{document}
