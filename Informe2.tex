\documentclass[a4paper,10pt]{article}
\usepackage[left=2cm,top=3cm,right=2cm,bottom=1cm,head=2.0cm,includefoot]{geometry}
\usepackage[utf8]{inputenc}
\usepackage[spanish]{babel}
\usepackage{comment}
\usepackage{fancyhdr}
\usepackage[T1]{fontenc}
\usepackage{graphicx}
\usepackage{bookman}
\usepackage{amsmath}
\usepackage{color}
\usepackage{listings}
\usepackage{longtable}
\usepackage{moreverb}
\usepackage{booktabs}
\usepackage{multirow}
\usepackage{ulem}
\usepackage[pdfborder={0 0 0 0}]{hyperref}
\usepackage{fixltx2e}
\usepackage{float}
\usepackage{wrapfig}
\usepackage{soul}
\usepackage{t1enc}
\usepackage{textcomp}
\usepackage{marvosym}
\usepackage{wasysym}
\usepackage{latexsym}
\usepackage{amssymb}
\usepackage{microtype}
\usepackage{hyperref}
\usepackage{pdfpages} % to import PDF pages
\tolerance=1000

\begin{document}

\pagestyle{fancy}
\chead{75.15 - Base de Datos}
\lhead{\includegraphics[width=1.5cm]{./logo1.png}}
\lfoot{Grupo 2}
\rfoot{$2^{do}$ Cuatrimestre 2010}

%\tableofcontents
%\newpage

\section*{Consultas SQL}

\subsection*{Enunciado}

 Para realizar las consultas se deben utilizar , sin excepción , las
  tablas de la base de datos del link precitado\footnote{Para acceder a la base de datos:  \href{http://apex.oracle.com/pls/otn/f?p=26093:1}{http://apex.oracle.com/pls/otn/f?p=26093:1} con usuario y password: test. }.  Utilizando
  únicamente las sentencias y cláusulas especificadas en la cartilla
  de sintaxis SQL incluída en el apunte, obtener:

\begin{enumerate}
\item Todos los programas que no se emiten los fines de semana.
\item El o los días de la semana en que el anunciante \textbf{“Juan Uncio”},
     anuncie su mayor cantidad diaria de anuncios.
\item La razón social de los anunciantes que anuncian en la mayor
     cantidad de programas distintos.
\item El nombre de los programas grabados que se emitan en promedio más
     veces por día los fines de semana que los días laborales (lunes a
     viernes).
\item El día, hora y número de tanda de las tandas que tengan una
     cantidad de anuncios mayor que el promedio de anuncios por tanda
     del mismo día.
\item Para cada día, mostrar el día, la hora y el nombre del programa
     que se emite en el espacio con la mayor cantidad de anuncios
     distintos de cada día.
\item El nombre de los programas en vivo que se emiten en todos los
     días hábiles (lunes a viernes).
\item Los días en que se ocupan todos los estudios.
\item La razón social de los anunciantes que tienen al menos un anuncio
     de sus productos en las emisiones de todos los programas del
     productor \textbf{“ROMUALDO”}.
\item El nombre de los programas que se emiten en todos los días en
      que al menos una vez sean utilizados todos los móviles por
      cualquiera de los programas emitidos en ese día.
\end{enumerate}
\newpage

\subsection*{Resoluci\'on}

\begin{enumerate}


 \item No se que escribir aca =P
\begin{tabbing}
 \textbf{SELECT} \= p.COD\_PROG, p.NOMBRE\\
 \textbf{FROM} \=PROGRAMA p\\
 \textbf{WHERE NOT EXISTS} (\\
 \> \textbf{SELECT} \= *\\
 \>  \textbf{FROM} \= PROGRAMA p2, ESPACIO e\\
 \>   \textbf{WHERE} \= p2.COD\_PROG = P.COD\_PROG \textbf{AND} p.COD\_PROG = e.COD\_PROG \\ \>   \=\textbf{AND} (e.DIA = 'DOM'\textbf{OR} e.DIA = 'SAB') \\)\\

\end{tabbing}

 \item siguiente
lala

\end{enumerate}


\end{document}
